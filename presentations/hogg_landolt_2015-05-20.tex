\documentclass[pdftex]{beamer}
%\input{vc}
% 1.77778 is the ratio of 16 to 9
\setlength{\paperheight}{3.5in}
\setlength{\paperwidth}{1.77778\paperheight}
% 1.33333 is the ratio of 4 to 3
%\setlength{\paperheight}{4.0in}
%\setlength{\paperwidth}{1.33333\paperheight}
\setlength{\textwidth}{0.85\paperwidth}
% import the next thing *after* the papersize
\input{./hogg_presentation} % hogg standard colors

\title{The theory and practice of self-calibration}
\author[David W. Hogg (NYU)]{David W. Hogg \\
  \textsl{\small Center for Cosmology and Particle Physics,
                 New York University} \\
  \textsl{\small Center for Data Science,
                 New York University} \\
  \textsl{\small Max-Planck-Insitut f\"ur Astronomie, Heidelberg}}
\date{2015 May 20}

\newcommand{\conclusions}{%
\begin{frame}
  \frametitle{summary}
  \begin{itemize}
  \item \sdss\ switched from a standards-based calibration to self-calibration.
    \begin{itemize}
    \item no use of standard stars whatsoever (except\ldots)
    \item extremely successful
    \item required certain kinds of non-trivial observational redundancy
    \end{itemize}
  \item New surveys plan single-pass wide-field imaging surveys.
    \begin{itemize}
    \item \euclid, \wfirst
    \item imaging with a repeated dither pattern on a grid is a bad, bad idea
    \item patterns far better for calibration can be equally efficient
    \end{itemize}
  \item Ideas from causal inference bring new, far crazier ideas to self-calibration.
    \begin{itemize}
    \item surveys that never repeat and have no redundancy
    \item \kepler\ and \kt\ lightcurves
    \item \wfc\ flat-field
    \end{itemize}
  \end{itemize}
\end{frame}}

\begin{document}

\conclusions

\begin{frame}
  \titlepage
\end{frame}

\begin{frame}
  \frametitle{Potted \sdss\ calibration history}
\end{frame}

\begin{frame}
  \frametitle{\sdss\ self-calibration {\footnotesize (Padmanabhan \etal, arXiv:XXX)}}
\end{frame}

\begin{frame}
  \frametitle{\sdss\ self-calibration {\footnotesize (Padmanabhan \etal, arXiv:XXX)}}
\end{frame}

\begin{frame}
  \frametitle{\sdss\ self-calibration: comments}
  \begin{itemize}
  \item millions of star parameters, tens of thousands of calibration parameters
    \begin{itemize}
    \item nonetheless over-constrained
    \item made convex by approximation
    \end{itemize}
  \item still used AB zeropoints of BD+17~4708
  \item fundamentally about \emph{obtaining consistency}
    \begin{itemize}
    \item requires same star to hit different detector positions
    \item appropriate redundancy
    \end{itemize}
  \end{itemize}
\end{frame}

\begin{frame}
  \frametitle{imaging survey design {\footnotesize (Holmes, Hogg, Rix, arXiv:1203.6255)}}
  ~\hfill%
  \includegraphics<1>[height=\figureheight]{./1203.6255v2/simple_surveys.pdf}
  \includegraphics<2>[height=\figureheight]{./1203.6255v2/A_5000_ff.png}
  \includegraphics<3>[height=\figureheight]{./1203.6255v2/B_1114_ff.png}
  \includegraphics<4>[height=\figureheight]{./1203.6255v2/C_24_ff.png}
  \includegraphics<5>[height=\figureheight]{./1203.6255v2/D_11_ff.png}
  \includegraphics<6>[height=\figureheight]{./1203.6255v2/simple_surveys.pdf}
\end{frame}

\begin{frame}
  \frametitle{imaging survey design: comments}
  \begin{itemize}
  \item recommend off-by-one and psuedo-random coverage patterns
    \begin{itemize}
    \item move the same star to different detector positions
    \item (give more uniform coverage than grids)
    \item (don't take additional time)
    \end{itemize}
  \item challenges are social
  \end{itemize}
\end{frame}

\begin{frame}
  \frametitle{radical self-calibration}
\end{frame}

\begin{frame}
  \frametitle{the NASA \kepler\ \& \kt\ Missions}
  \begin{itemize}
  \item deliberately try to fix stars in detector coordinates
    \begin{itemize}
    \item no redundancy (of the kind we want)
    \item pessimal for self-calibration
    \end{itemize}
  \item most detector pixels never get telemetered down
    \begin{itemize}
    \item very sparse information on the focal plane
    \end{itemize}
  \item instrument and flat-field model not good enough
    \begin{itemize}
    \item routinely performing photometry with \emph{precision} around $10^5$
    \end{itemize}
  \end{itemize}
\end{frame}

\begin{frame}
  \frametitle{radical self-calibration {\footnotesize (Wang \etal, in preparation)}}
  \begin{itemize}
  \item if a star's behavior can be predicted confidently by
    \emph{other stars} then that behavior must be being imprinted by
    the spacecraft
  \item using these ideas to separate spacecraft-induced from
    intrinsic stellar variability
  \end{itemize}
\end{frame}

\begin{frame}
  \frametitle{\kt\ calibration {\footnotesize (Foreman-Mackey \etal, arXiv:1502.04715)}}
  ~\hfill
  \includegraphics<1>[trim=100 100 100 100, clip, height=\figureheight]{brownbag/brownbagp10.pdf}
  \includegraphics<2>[trim=100 100 100 100, clip, height=\figureheight]{brownbag/brownbagp14.pdf}
  \includegraphics<3>[trim=100 100 100 100, clip, height=\figureheight]{brownbag/brownbagp15.pdf}
  \includegraphics<4>[height=0.9\figureheight]{1502.04715/figures-corr.pdf}
\end{frame}

\begin{frame}
  \frametitle{NASA \hst\ \wfc\ imaging {\footnotesize (Vakili, Fadely, Hogg, in prep)}}
  \begin{itemize}
  \item could we calibrate with no repeat observations of any kind?
  \item images have to be generated by mixtures of point sources!
  \item simultaneously model PSF and every scene
  \item residuals that are consistent in detector position are flat-field (or vignetting) issues.
    \begin{itemize}
    \item a calibration of \hst\ that makes no use whatsoever of any calibration observations
    \item relies on having many, many observations
    \item (currently vapor-ware)
    \end{itemize}
  \end{itemize}
\end{frame}

\conclusions

\end{document}

\begin{frame}
  \frametitle{the NASA \kepler\ Mission}
  \begin{itemize}
  \item stared at 150,000 stars for 4 years (30-min cadence)
    \begin{itemize}
    \item 70,000 measurements per star
    \item all data completely public
    \end{itemize}
  \item looking for exoplanet transit signals
  \item found \emph{thousands} of exoplanets (candidates)
  \end{itemize}
\end{frame}

\begin{frame}
  \frametitle{the NASA \kepler\ Mission}
  \begin{itemize}
  \item stared at 150,000 stars for 4 years (30-min cadence)
    \begin{itemize}
    \item 70,000 measurements per star
    \item all data completely public
    \end{itemize}
  \item looking for exoplanet transit signals
  \item found \emph{thousands} of exoplanets (candidates)
  \end{itemize}
\end{frame}

\begin{frame}
  \frametitle{the NASA \kepler\ Mission}
  \begin{itemize}
  \item stared at 150,000 stars for 4 years (30-min cadence)
    \begin{itemize}
    \item 70,000 measurements per star
    \item all data completely public
    \end{itemize}
  \item looking for exoplanet transit signals
  \item found \emph{thousands} of exoplanets (candidates)
  \end{itemize}
\end{frame}

\conclusions

\end{document}

\begin{frame}
  \frametitle{the NASA \kepler\ Mission}
  \begin{itemize}
  \item stared at 150,000 stars for 4 years (30-min cadence)
    \begin{itemize}
    \item 70,000 measurements per star
    \item all data completely public
    \end{itemize}
  \item looking for exoplanet transit signals
  \item found \emph{thousands} of exoplanets (candidates)
  \end{itemize}
\end{frame}

\conclusions

\end{document}
