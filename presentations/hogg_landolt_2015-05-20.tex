\documentclass[pdftex]{beamer}
%\input{vc}
% 1.77778 is the ratio of 16 to 9
\setlength{\paperheight}{3.5in}
\setlength{\paperwidth}{1.77778\paperheight}
% 1.33333 is the ratio of 4 to 3
%\setlength{\paperheight}{4.0in}
%\setlength{\paperwidth}{1.33333\paperheight}
\setlength{\textwidth}{0.85\paperwidth}
% import the next thing *after* the papersize
\input{./hogg_presentation} % hogg standard colors

\title{The theory and practice of self-calibration}
\author[David W. Hogg (NYU)]{David W. Hogg \\
  \textsl{\small Center for Cosmology and Particle Physics,
                 New York University} \\
  \textsl{\small Center for Data Science,
                 New York University} \\
  \textsl{\small Max-Planck-Insitut f\"ur Astronomie, Heidelberg}}
\date{2015 May 20}

\newcommand{\conclusions}{%
\begin{frame}
  \frametitle{summary}
  \begin{itemize}
  \item exoplanet research is shaped by engineering and data analysis challenges
    \begin{itemize}
    \item I focus on these issues
    \end{itemize}
  \item \emph{search} involves extracting tiny, sparse signals from (huge) data sets
    \begin{itemize}
    \item noise modeling and marginalization of systematics
    \end{itemize}
  \item \emph{characterization} leaves us with noisy individual-planet measurements
  \item \emph{population inferences} require expensive noise propagation
    \begin{itemize}
    \item hierarchical probabilistic inference
    \end{itemize}
  \item Earth-like planets are plentiful in our Galaxy
    \begin{itemize}
    \item a percent (or more) of Sun-like stars host Earth-like planets
    \item the Solar System is not obviously typical in any way
    \item little is known about Jupiter-like planets
    \end{itemize}
  \end{itemize}
\end{frame}}

\begin{document}

\conclusions

\begin{frame}
  \titlepage
\end{frame}

\begin{frame}
  \frametitle{the NASA \kepler\ Mission}
  \begin{itemize}
  \item stared at 150,000 stars for 4 years (30-min cadence)
    \begin{itemize}
    \item 70,000 measurements per star
    \item all data completely public
    \end{itemize}
  \item looking for exoplanet transit signals
  \item found \emph{thousands} of exoplanets (candidates)
  \end{itemize}
\end{frame}

\begin{frame}
  \frametitle{the NASA \kepler\ Mission}
  \begin{itemize}
  \item stared at 150,000 stars for 4 years (30-min cadence)
    \begin{itemize}
    \item 70,000 measurements per star
    \item all data completely public
    \end{itemize}
  \item looking for exoplanet transit signals
  \item found \emph{thousands} of exoplanets (candidates)
  \end{itemize}
\end{frame}

\begin{frame}
  \frametitle{the NASA \kepler\ Mission}
  \begin{itemize}
  \item stared at 150,000 stars for 4 years (30-min cadence)
    \begin{itemize}
    \item 70,000 measurements per star
    \item all data completely public
    \end{itemize}
  \item looking for exoplanet transit signals
  \item found \emph{thousands} of exoplanets (candidates)
  \end{itemize}
\end{frame}

\begin{frame}
  \frametitle{the NASA \kepler\ Mission}
  \begin{itemize}
  \item stared at 150,000 stars for 4 years (30-min cadence)
    \begin{itemize}
    \item 70,000 measurements per star
    \item all data completely public
    \end{itemize}
  \item looking for exoplanet transit signals
  \item found \emph{thousands} of exoplanets (candidates)
  \end{itemize}
\end{frame}

\end{document}

\begin{frame}
  \frametitle{the NASA \kepler\ Mission}
  \begin{itemize}
  \item stared at 150,000 stars for 4 years (30-min cadence)
    \begin{itemize}
    \item 70,000 measurements per star
    \item all data completely public
    \end{itemize}
  \item looking for exoplanet transit signals
  \item found \emph{thousands} of exoplanets (candidates)
  \end{itemize}
\end{frame}

\conclusions

\end{document}
